% beamer-nomi.tex
% template for presentation while officialy make presentation
% with JAIST as institution

% compilation:
% pdflatex filename.tex # but all path inside should be changed

\pdfminorversion=4
\documentclass{beamer}
\usepackage{animate} % for animation
\usepackage{array,multirow,graphicx}
\usepackage{algorithm,algorithmic}
\usepackage{verbatim}
\graphicspath{{/media/bagustris/atmaja/mypaper/2018/avec2018/}{./fig/}} 
\setbeamertemplate{caption}[numbered]

\usetheme[pageofpages=of,% String used between the current page and the
                         % total page count.
          alternativetitlepage=true,% Use the fancy title page.
          titleline=true,
          titlepagelogo=/media/bagustris/bagus/Pictures/logo/jaist_logo-3.png
%          titlepagelogo=fig/jaist_logo.png
          ]{Torino}
          % change /beamerinnerthemefancy.sty to resize the logo
\usecolortheme{freewilly}

\author{Bagus Tris Atmaja \\ 
        bagus@jaist.ac.jp}
\title{\LARGE{Beamer Nomi: \vspace{5pt}
            \newline A beamer template for JAIST }}
\institute{AIS-Lab \\ School of Information Science \\ JAIST}
%\date{October 29, 2018}

% The log drawn in the upper right corner.
\logo{\includegraphics[height=0.11\paperheight]{/media/bagustris/bagus/Pictures/logo/jaist_logo-1.png}}

\begin{document}

\begin{frame}[t,plain]
\titlepage
\end{frame}


\begin{frame}[t, fragile]{Motivation}
\begin{itemize}
  \item First motivation, \verb|within \itemize|
  \item Second motivation, \verb|with \item|
\end{itemize}
\end{frame}

\begin{frame}{Problem}
\begin{itemize}
\item   Given a set of (cross-cultural) dataset (train and devel) with its label, 
        how to predict emotion dimension on test data?
\item   How to build multimodal emotion recogniton with LSTM network?
\item   How feature selection can improve AER? \\
\vspace{5pt}
Purpose: \\
\item Build multimodal emotion recognition using feature selection and LSTM, compare 
with baseline sytem (without feature selection)
\end{itemize}
\end{frame}

%\begin{frame}[t, fragile]{Research Proposal and Motivation}
%	\begin{itemize}
%		\item Main purpose: \\
%				Robust speech emotion recognition system by utilizing acoustic 
%						and text feature.
%		\item Steps: \\
%		\begin{enumerate}
%			\item Build speech emotion recognition by using LSTM-RNN\footnotemark ~algorithm from sequential 
%			acoustic feature and/or CNN\footnotemark ~algorithm from spectrograms feature.
%			\item Build text emotion recognition by using new ANEW (Affective 
%			norms for English word) analysis and compare it to machine learning method.
%			\item Integrate step 1 and 2 to improve speech emotion performance.
%		\end{enumerate}
%	\end{itemize}
%\centering 
%\footnotetext[1]{\tiny Long short term memory - recurrent neural network}
%\footnotetext[2]{\tiny Convolutional neural network}
%\end{frame}

\begin{frame}[t, fragile]{Example of Table}
\begin{itemize}
\item SEWA dataset is used, consisting audiovisual spontaneous behaviors of 
      participant recorded \emph{in-the-wild}.
\item Annotation (labels) is available for the emotional dimensions arousal and valence, and a third dimension describing liking (or sentiment), by 6 (German) or 5 (Hungarian) native speakers.
\end{itemize}
\begin{table}[ht]
  \caption{Number of instance for each partition of recordings}
  \label{tab:instance}
  \begin{tabular}{c c c c c}
    \hline
    Partitions & German & Hungarian & Labels & Total \\
    \hline
    Training	&	34	&	-	& $\checkmark$	&	34	\\
	Development	&	14	&	-	& $\checkmark$	&	14	\\
	Testing		&	16	&	66	&	-			&	82	\\
	Total		&	64	&	66	&	130			&	130	\\
	\hline
	\end{tabular}

\end{table}

\end{frame}

\begin{frame}[t, fragile]{Example of footnote as reference}
\begin{itemize}
\item Use \verb|\footnotetext| for Reference.
\item Use \verb|\footnotemark| to mark.
\item Available features:
\end{itemize}
\begin{table}
  \caption{ Number of audio and video features}
  \label{tab:numfeatures}
  \begin{tabular}{c c c}
    \hline
    Model & LLDs & Bag of words \\
    \hline
    Audio	&	23 eGeMAPS	& 100	\\
			&	39 MFCCs	& 100	\\
	Visual	&	17 FAUs	&	100	\\
	\hline
	\end{tabular}
\end{table}
\footnotetext{\centering eGeMAPS: Geneva Minimalistic Acoustic Parameter Set, Eyben et. al., 2013 \\
              MFCCs: 1 to 13, inculding deltas and deltas-deltas, totally 39 \\
              FAU: Facial Action Units}
\end{frame}

\begin{frame}[t, fragile]{Example of footnote}
\begin{itemize}
\item Use \verb|\centering| for centering footnote, sometimes looks ugly.
\item Use \verb|\tiny| to make footnote text tiny\footnotemark.
\item Line above footnote is automatic.
\item figure within \verb|\item| doesn't need \verb|\centering|
\includegraphics[width=3.5in]{/media/bagustris/atmaja/s3/kenkyu/pic/aco.png}
\end{itemize}
\footnotetext{\tiny F. Ringeval et al., “AVEC 2017-Real-life Depression, and Affect Recognition Workshop and Challenge,” 2017.}
\end{frame}

\begin{frame}[t, fragile]{Example of Table: plain latex}
\begin{center}
\includegraphics[width=3in]{/media/bagustris/atmaja/mypaper/2018/avec2018/pict/svm3.png}
\begin{table}
  \caption{Result of Optimal Set algorithm using visual features (BoVW FAUs)}
  \label{tab:val}
  \begin{tabular}{c c c c c c}
    \hline
    Valence FID	& CCC & Arousal FID	& CCC	& Liking FID &	CCC \\
    \hline
	VI31	&	0.192	&	VI31	&	0.274	&	VI99	&	0.098	\\
	VI17	&	0.170	&	VI17	&	0.239	&	VI74	&	0.055	\\
	VI60	&	0.153	&	VI62	&	0.187	&	VI41	&	0.041	\\
	VI20	&	0.147	&	VI20	&	0.177	&	VI40	&	0.027	\\
	VI35	&	0.125	&	VI60	&	0.164	&	VI12	&	0.019	\\
	\hline
	\end{tabular}
\end{table}
\end{center}
\end{frame}

\begin{frame}[t, fragile]{Example algorithm, need algorithmic package}
 \begin{algorithmic}[1]
    \STATE  Input gold standard values for emotion 
        dimension $ED_i$ for development partition.
    \STATE Input features sorted by abs(CCC):
        \\ $f_1, f_2, f_3,...,f_n$ 
    \STATE Input impact $CCC_1$  of  $f_1$ 
    \STATE $Optimal\_Set = \{f_1\}, CCC\_Optimal = CCC_1$
    \FOR {$f_j$ in {$f_1, f_2, f_3, ..., f_n$}}
  \STATE $CCC_j$=CCC(Predict ([$Optimal\_Set, f_j$]))
  \IF {$CCC_j > CCC\_Optimal$ }
  \STATE $CCC\_Optimal = CCC_j$        
        \\ $Optimal\_Set = [Optimal\_Set, f_j]$
  \ENDIF
  \ENDFOR
 \RETURN $Optimal\_Set, CCC\_Optimal$
 \end{algorithmic} 
\end{frame}

%\begin{frame}[t, fragile]{Text Features: bag-of-word}
%\begin{itemize}
%\item a bag-of-words feature representation based on the transcription of the speech are generated with openXBOW\footnotemark ~and used as additional features. 
%\item The dictionary for these textual features is learnt from the training partition taking only the terms with at least two occurrences into account. This results in a dictionary of "n-number" words, where only unigrams are considered. 
%\item In total, the bag-of-text-words (BoTW) features contain "n-number" features. This text features will be trained along with acoustic feature fed to LSTM network.
%\end{itemize}
%\footnotetext[5]{\tiny Maximilian Schmitt and Bj¨orn W. Schuller. 2016. openXBOW – Introducing the Passau open-source crossmodal Bag-of-Words toolkit. preprint arXiv:1605.06778 (2016).}
%\end{frame}

\begin{frame}[t, fragile]{Method: LSTM network}
\begin{itemize}
\item Using LSTM algorithm to train valence, arousal and liking 
      from German language to predict its dimension from different 
      number of acoustic and visual features.
\item LSTM network can model the context (VAD value) while
      insensitive to outliers\footnotemark
\item Network architecture:\\
\includegraphics[width=4in]{/media/bagustris/atmaja/mypaper/2018/avec2018/pict/system.eps}
\end{itemize}
\footnotetext{\tiny P. Tzirakis, G. Trigeorgis, M. A. Nicolaou, B. Schuller, and S. Zafeiriou, “End-to-End Multimodal Emotion Recognition using Deep Neural Networks,” vol. 14, no. 8, pp. 1–9, 2017.}
\end{frame}

\begin{frame}[t, fragile]{Use column two divide into 2 sides}
\begin{columns}
\column{.4\textwidth} 
\begin{table}
  \caption{Parameters in LSTM network}
  \label{tab:lstm}
  \begin{tabular}{ll}
    \hline
    Parameter & Value \\
    \hline
    batch size    & 34       \\
    learning rate & 0.001    \\
    num iter      & 50      \\
    num units 1   & 128     \\
    num units 2   & 64      \\
    bidirectional & False    \\
    dropout       & 0.2     \\
    \hline
\end{tabular}
\end{table}
\column{.6\textwidth} 
\includegraphics[width=2.7in]{/media/bagustris/atmaja/mypaper/2018/avec2018/pict/lstm.eps}
\end{columns}
\end{frame}

\begin{frame}[t, fragile]{Example of Equations}
We use the following objective function to measure the performance.
$x$ is each VAD (valence, arousal, dominance) score from dataset, 
and $y$ is predicted each VAD score from our algorithm.

\begin{itemize}
\item Concordance Correlation Coefficient (CCC): 
\begin{equation}
	CCC_i = \dfrac{2\rho \sigma_x \sigma_y}
			{\sigma_x^2 + \sigma_y^2 + (\mu_x - \mu_y)^2}
\label{eq:ccc}
\end{equation}

The average of CCC for the three-emotion dimension is used as a measure for the performance of the whole system; 
this measure is defined by the following equation.  

\begin{equation}
	CCC_{avg} = \dfrac {\Sigma~CCC_i}{n}
\end{equation}

\noindent
where $n$ is number of $i$ dimension, i.e. 3 (valence, arousal, liking).
\end{itemize}
\end{frame}

\begin{frame}[t, fragile]{Example of full table}
\begin{table}[ht]
\centering
\caption{Evaluation results using mono-language case by using the development partition form German language using selected features from audio-video modalities }
\begin{tabular}{c c c c c}
\hline
Features Set	&	Arousal	&	Valence	&	Liking	&	Average \\
\hline
Baseline (100) 	&	0.552	&	0.563	&	0.238	&	0.451 \\
Selected (6)	&	0.641	&	0.636	&	0.278	&	0.518 \\
Selected (10)	&	0.660	&	0.620	&	0.298	&	0.526 \\
Selected (15)	&	0.622	&	0.623	&	\textbf{0.314}	&	0.520 \\
Selected (20)	&	0.616	&	0.596	&	0.299	&	0.504 \\
Optimal Set	&		\textbf{0.678}	&		\textbf{0.654}	&	0.304	&	\textbf{0.545} \\
\hline
\end{tabular}
\label{tab:DevelDE}
\end{table}
\end{frame}

\begin{frame}[t, fragile]{Result: Testing Data}
\includegraphics[width=4.5in]{/media/bagustris/atmaja/mypaper/2018/avec2018/pict/compareCCC.png}
\end{frame}

\begin{frame}[t, fragile]{Conclusion}
\begin{itemize}
	\item The use of deep learning technique (LSTM-RNN/CNN) for dimensional 
	      speech emotional recognition from multimodal feauture has been presented.
	\item The number of dominant feature extracted from bag-of-acoustic-words
	      (BoAW) and bag-of-text-words (BoTW) that contributes significantly to 
	      speech emotion recognition
	      performance by feature selection algorithm.
	\item The result shows promising result on development data, slightly improvement on 
	      testing data, but poor performance on cross-cultural test data, due to 
	      limitation of dataset.
\end{itemize}
\end{frame}



\begin{frame}[t, fragile]{Remaining Problem/Future Works}
\begin{itemize}
\item Use total CCC instead of averaged CCC as overall accurate prediction is desired.
\item Compare Acoustic feature only to Audio/Video-based.
\item Implement the similar scenario for IEMOCAP dataset.
\end{itemize}
\end{frame}
\end{document}

